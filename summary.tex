\documentclass[10pt,landscape]{article}
\usepackage{multicol}
\usepackage{calc}
\usepackage{ifthen}
\usepackage[a4paper,landscape]{geometry}
\usepackage{hyperref}
\usepackage{tcolorbox}

% This sets page margins to .5 inch if using letter paper, and to 1cm
% if using A4 paper. (This probably isn't strictly necessary.)
% If using another size paper, use default 1cm margins.
\ifthenelse{\lengthtest { \paperwidth = 11in}}
	{ \geometry{top=.5in,left=.5in,right=.5in,bottom=.5in} }
	{\ifthenelse{ \lengthtest{ \paperwidth = 297mm}}
		{\geometry{top=1cm,left=1cm,right=1cm,bottom=1cm} }
		{\geometry{top=1cm,left=1cm,right=1cm,bottom=1cm} }
	}

% Turn off header and footer
\pagestyle{empty}


% Redefine section commands to use less space
\makeatletter
\renewcommand{\section}{\@startsection{section}{1}{0mm}%
                                {-1ex plus -.5ex minus -.2ex}%
                                {0.5ex plus .2ex}%x
                                {\normalfont\large\bfseries}}
\renewcommand{\subsection}{\@startsection{subsection}{2}{0mm}%
                                {-1explus -.5ex minus -.2ex}%
                                {0.5ex plus .2ex}%
                                {\normalfont\normalsize\bfseries}}
\renewcommand{\subsubsection}{\@startsection{subsubsection}{3}{0mm}%
                                {-1ex plus -.5ex minus -.2ex}%
                                {1ex plus .2ex}%
                                {\normalfont\small\bfseries}}
\makeatother

% Define BibTeX command
\def\BibTeX{{\rm B\kern-.05em{\sc i\kern-.025em b}\kern-.08em
    T\kern-.1667em\lower.7ex\hbox{E}\kern-.125emX}}

% Don't print section numbers
\setcounter{secnumdepth}{0}


\setlength{\parindent}{0pt}
\setlength{\parskip}{0pt plus 0.5ex}


% Colored boxes
% new tcolorbox environment
% #1: tcolorbox options
% #2: color
% #3: box title
\newtcolorbox{textbox}[3][]
{
  colframe = #2!25,
  colback  = #2!10,
  coltitle = #2!20!black,
  title    = #3,
  #1,
}


% -----------------------------------------------------------------------

\begin{document}

\raggedright
\footnotesize
\begin{multicols}{3}


% multicol parameters
% These lengths are set only within the two main columns
%\setlength{\columnseprule}{0.25pt}
\setlength{\premulticols}{1pt}
\setlength{\postmulticols}{1pt}
\setlength{\multicolsep}{1pt}
\setlength{\columnsep}{2pt}

\begin{center}
     \Large{CCNA Summary} \\
\end{center}

\section{CCNA Routing \& Switching \textit{200-120}}

\subsection{Understanding Networks and their Building Blocks}
\begin{textbox}{gray}{TODO}
	This chapter is not yet complete!
\end{textbox}

\subsection{IP Addressing and Subnets}
\begin{textbox}{gray}{TODO}
	This chapter is not yet complete!
\end{textbox}

\subsection{Introduction to Cisco Routers, Switches and IOS}
\begin{textbox}{gray}{TODO}
	This chapter is not yet complete!
\end{textbox}

\subsection{Introduction to IP Routing}
\begin{textbox}{gray}{TODO}
	This chapter is not yet complete!
\end{textbox}

\subsection{Routing Protocols}
\begin{textbox}{gray}{TODO}
	This chapter is not yet complete!
\end{textbox}

\subsection{Switching and Spanning Tree Protocol}
\begin{textbox}{gray}{TODO}
	This chapter is not yet complete!
\end{textbox}

\subsection{VLANs and VTP}
\begin{textbox}{gray}{TODO}
	This chapter is not yet complete!
\end{textbox}

\subsection{Network Security}
\begin{textbox}{gray}{TODO}
	This chapter is not yet complete!
\end{textbox}

\subsection{Access Lists}
\begin{textbox}{gray}{TODO}
	This chapter is not yet complete!
\end{textbox}

\subsection{Network Address Translation (NAT)}
\begin{textbox}{gray}{TODO}
	This chapter is not yet complete!
\end{textbox}

\subsection{Wide Area Networks}
\begin{textbox}{gray}{TODO}
	This chapter is not yet complete!
\end{textbox}

\subsection{Virtual Private Networks}
\begin{textbox}{gray}{TODO}
	This chapter is not yet complete!
\end{textbox}

\subsection{IPv6}
\subsubsection{IPv6 Introduction}
\paragraph{}
Due to the shortcomings of IPv4, the Internet Protocol version 6 (IPv6) has been created.
The main reason for migratig TCP/IP networks from IPv4 to IPv6 is the avaiable address space.
While IPv4 uses a 32-bit address, IPv6 uses a 128-bit address.
The change from IPv4 to IPv6 also impacts other protocols as well (\textit{OSPFv3, EIGRPv6, etc.}).
\paragraph{}
Just like IPv4, the main objective of IPv6 is to enable devices to forward packets through multiple routers so they arrive at the correct destination.
However, IPv6 contains a number of differences over IPv4:
\begin{itemize}
	\item Larger address space;
	\item Auto-configuration;
	\item The IPv6 header is \textit{not} similar to the IPv4 header;
	\item Extension headers/options;
	\item Authentication and privacy;
	\item Flow labels (\textit{QoS}).
\end{itemize}
\paragraph{}
There are thee types of IPv6 addresses:
\begin{description}
	\item[Unicast] Unique address for each interface.
	\item[Anycast] Multiple interfaces, packets are send to one (\textit{nearest}).
	\item[Multicast] Multiple interfaces, packets are send to all.
\end{description}
\begin{textbox}{green}{Key Concept}
	IPv6 broadcast addresses are special case of multicast addresses.
\end{textbox}
\paragraph{}
An IPv6 address is a 128-bit value, displayed as 8 groups of 4 hexadecimal digits.
For example: \verb!2001:0DB8:0000:0000:0006:0600:300D:527B!.
Leading zeros can be left out: \verb!2001:DB8:0:0:6:600:300D:527B!, one or more adjecent groups
of 16 bit of zeros can be replaced with the \verb!::! symbol (\textit{once!}): \verb!2001:DB8::6:600:300D:527B!.
\paragraph{}
IPv6 provides tow similar options for unicast addressing:
\begin{description}
	\item[Global Unicast] Similar to public IPv4 addresses. These addresses are allocated by the IANA. Each company is assigned a unique IPv6 address block called a \textit{global routing prefix}. Global Unicast addresses make up the majority of IPv6 addresses.
	\item[Unique Local] Similar to private IPv4 addresses. Can by used by when behind a IPv6 NAT and in networks that aren't connected to the internet.
\end{description}
\paragraph{}
IPv6 addresses can be identified by the initial bits of the address:
\begin{tabular}{@{}l@{}l@{}l@{}}
\textit{Address Type} & \textit{Binary Prefix} & \textit{IPv6 Notation} \\
Unspecified                       & %
        \verb!000…0 (128 bits)!   & %
        \verb!::/128!             \\
Loopback                          & %
        \verb!000…1 (128 bits)!   & %
        \verb!::1/128!            \\
Multicast                         & %
        \verb!1111 1111!          & %
        \verb!FF00::/8!           \\
Link-Local Unicast                & %
        \verb!1111 1110 10!       & %
        \verb!FF80::/10!          \\
Global Unicast                    & %
        \textit{everthing else}   & %
        \textit{everthing else}
\end{tabular}

\subsubsection{IPv6 Address Configuration}
\begin{textbox}{gray}{TODO}
	This chapter is not yet complete!
\end{textbox}
\subsubsection{OSPF version 3}
\begin{textbox}{gray}{TODO}
	This chapter is not yet complete!
\end{textbox}
\subsubsection{EIGRP for IPv6}
\begin{textbox}{gray}{TODO}
	This chapter is not yet complete!
\end{textbox}

\subsection{IP Services}
\begin{textbox}{gray}{TODO}
	This chapter is not yet complete!
\end{textbox}

\rule{0.3\linewidth}{0.25pt}
\scriptsize

\href{https://github.com/roaldnefs/ccna}{https://github.com/roaldnefs/ccna}

\end{multicols}
\end{document}
